% Preamble
\documentclass[12pt, aspectration=1610]{beamer}
\usetheme{Copenhagen}
\usecolortheme{beaver}

\setbeamertemplate{navigation symbols}{}

% Packages
\usepackage{amsmath}
\usepackage{listings}
\usepackage{xcolor}
\usepackage{blkarray}
\usepackage{graphicx}
\graphicspath{ {./images/} }
\usepackage{hyperref}

\definecolor{codegreen}{rgb}{0,0.6,0}
\definecolor{codegray}{rgb}{0.5,0.5,0.5}
\definecolor{codepurple}{rgb}{0.58,0,0.82}
\definecolor{backcolour}{rgb}{0.95,0.95,0.92}

\title{Processes and Threads}
\author{Kai Striega}
\date{2025-12-13}

% Document
\begin{document}

    \maketitle

    \begin{frame}{Who am I?}
        \begin{itemize}
            \item Self-taught developer
            \item Work as a Senior Software Engineer at Endgame Analytics
            \item Volunteer in the Free and Open Source community
            \item Organised the Scientific Python Specialist Track at PyCon Australia
            \item SheCodes mentor since early 2022
        \end{itemize}
    \end{frame}

    \begin{frame}{Before I start}
        \begin{block}{Here be dragons}<2->
            This talk introduces advanced topics, quickly.
            The content is designed to stretch your understanding and to challenge you.
        \end{block}
        \begin{block}{Questions}<3->
            I am happy to take questions during the talk, feel free to ask if something doesn't make sense.
        \end{block}
    \end{frame}

    \begin{frame}{Expanding your technical toolbox}
        \begin{enumerate}
            \item As a programmer, you’re continually building up your technical toolbox
            \item This toolbox is the sum total of all the tools that you have available to solve any given problem
            \item Processes or threads provide another, extremely powerful tool to add to your toolbox
            \item Helps you understand what happens when a program executes
            \item ``Process vs Thread'' is also a common interview question
        \end{enumerate}
    \end{frame}

    \begin{frame}{Today's Analogy}
        \begin{itemize}
            \item Processes and Threads can be difficult to understand without context
            \item So I'll teach by using an analogy
            \item This analogy is cooking
        \end{itemize}
    \end{frame}

    \begin{frame}{Why is cooking relevant?}
        \begin{minipage}{0.55\textwidth}
            \begin{itemize}
                \item So far you've learnt to write simple Python programs
                \item You've learnt to combine data (ingredients) using code (a recipe)
                \item a.k.a.\ you've learnt to cook small meals by yourself
                \item But you need more to cook a meal than just ingredients and a recipe
            \end{itemize}
        \end{minipage}
        \begin{minipage}{0.4\textwidth}
            \includegraphics[width=\linewidth]{recipe}
        \end{minipage}
    \end{frame}

    \begin{frame}
        Processes
    \end{frame}

    \begin{frame}{A process is the kitchen}
        \begin{figure}
            \includegraphics[width=\textwidth]{kitchen}\label{fig:figure}
        \end{figure}
    \end{frame}

    \begin{frame}{A Processes }
        \begin{block}{Process}<2->
            An instance of a program in execution, together with its associated resources and execution context.
        \end{block}
        \begin{itemize}
            \item A process provides \textbf{resources} with which you can cook your meal
            \begin{itemize}
                \item Memory to store your variables and instructions
                \item CPU Time The OS schedules CPU time by processes
                \item System Resources give access to things like file handlers and network connections
            \end{itemize}
        \end{itemize}
    \end{frame}


    \begin{frame}{Viewing processes}
        \begin{itemize}
            \item On Linux/MacOS it is possible to view processes by using the $ps \text{  } u$ command
            \item  This gives us a lot of information:
            \begin{itemize}
                \item[User] Who owns the processes?
                \item[PID] What is the process ID?
                \item[\%CPU] How much CPU is the process using, as a percentage of total available?
                \item[\%MEM] How much memory is the process using, as a percentage of total available?
                \item[RSS] resident set size, the non-swapped physical memory that a task has used
                \item[STAT] process state.\ What state is the process currently in?
            \end{itemize}
        \end{itemize}
    \end{frame}

    \begin{frame}{Creating our own process}
        \begin{itemize}
        \item Let's say we have a very simple Python program:
        \end{itemize}

        \lstinputlisting{code/simple_process.py}

        \begin{itemize}
            \item We can now launch it and see some things:
            \begin{itemize}
                \item $ps \text{  } u \text{  } | \text{  } grep \text{  } python$
            \end{itemize}
        \end{itemize}
    \end{frame}

    \begin{frame}{Processes for parallelism}
        \begin{itemize}
            \item So far all your recipies do one thing at a time
            \item \ldots but computers can do many things at the same time
            \item This is analogous to having multiple kitchens to cook in
        \end{itemize}
        \vspace{5mm}
        \begin{minipage}{0.3\textwidth}
            \includegraphics[width=\linewidth]{kitchen}
        \end{minipage}
        \begin{minipage}{0.3\textwidth}
            \includegraphics[width=\linewidth]{kitchen}
        \end{minipage}
        \begin{minipage}{0.3\textwidth}
            \includegraphics[width=\linewidth]{kitchen}
        \end{minipage}
    \end{frame}

    \begin{frame}{Launching multiple processes}
        \begin{itemize}
            \item Naively we can launch Python multiple times by launching multiple terminals
            \item This works but is a hassle
            \item Instead we can use the inbuilt ``multiprocessing'' module
            \item The module allows us to launch a processes from another process
        \end{itemize}
    \end{frame}

    \begin{frame}{Creating processes from another process}
        \lstinputlisting{code/multiple_processes.py}
    \end{frame}

    \begin{frame}{Check-in quiz}
        \begin{enumerate}
            \item What is the output of this program?
            \item How long does this program take to execute?
        \end{enumerate}
    \end{frame}

    \begin{frame}{Summary of Processes}
        We’ve covered a lot so far.
        Let’s take a second to revise.
        We have:
        \begin{itemize}
            \item defined processes as the operating system’s representation of a running program
            \item seen that we can show processes using the ps command
            \item explained the output of (some) of the ps command’s outputs
            \item shown how to launch multiple processes, both through the terminal and via Python
            \item shown that using multiple processes can introduce parallelism, allowing us to run many functions at the same time
        \end{itemize}
    \end{frame}

    \begin{frame}
        Threads
    \end{frame}

    \begin{frame}{Threads}
        \begin{minipage}{0.6\textwidth}
            \begin{itemize}
                \item Processes build the whole kitchen
                \item \ldots \textbf{every time} we want to do parallelism
                \item This can be highly inefficient
                \item What about sharing resources within the kitchen?
                \item Threads are analogous to multiple cooks in the kitchen all utilising the same resources
            \end{itemize}
        \end{minipage}
        \begin{minipage}{0.35\textwidth}
            \includegraphics[width=\linewidth]{chef}
        \end{minipage}
    \end{frame}

    \begin{frame}{Many Threads}
        \includegraphics[width=\linewidth]{many_chefs}
    \end{frame}
    \begin{frame}{Creating threads from a process}
        \lstinputlisting{code/multiple_threads.py}
    \end{frame}

    \begin{frame}{Seeing threads from a process}
        \begin{itemize}
            \item Let's run the trusty $ps \text{  } u \text{  } | \text{  } grep \text{  } python$ command again
            \item<2-> What do we notice here?
            \begin{itemize}
                \item Only one process appears
                \item The amount of virtual memory is much higher
            \end{itemize}
        \end{itemize}
    \end{frame}

    \begin{frame}{What do we notice about threads}
        \begin{itemize}
            \item We need to use the $-T$ flag to see threads
            \item The $PID$ of each thread is the same
            \item This is because the thread runs \textit{inside} the process
            \item This is significant, threads share the resources provided by the process
        \end{itemize}
    \end{frame}

    \begin{frame}{How do threads and processes relate?}
        \includegraphics[height=\textheight]{kitchen}
    \end{frame}
    \begin{frame}{A kitchen must always have a chef}
        \includegraphics[width=\linewidth]{chef}
    \end{frame}
    \begin{frame}{You can have multiple chefs per kitchen}
        \includegraphics[width=\linewidth]{many_chefs}
    \end{frame}
    \begin{frame}{And you can have multiple kitchens}
        \begin{minipage}{0.3\textwidth}
            \includegraphics[width=\linewidth]{kitchen}
        \end{minipage}
        \begin{minipage}{0.3\textwidth}
            \includegraphics[width=\linewidth]{kitchen}
        \end{minipage}
        \begin{minipage}{0.3\textwidth}
            \includegraphics[width=\linewidth]{kitchen}
        \end{minipage}
    \end{frame}
    \begin{frame}{You can even have multiple chefs in multiple kitchens}
        \begin{minipage}{0.3\textwidth}
            \includegraphics[width=\linewidth]{many_chefs}
        \end{minipage}
        \begin{minipage}{0.3\textwidth}
            \includegraphics[width=\linewidth]{many_chefs}
        \end{minipage}
        \begin{minipage}{0.3\textwidth}
            \includegraphics[width=\linewidth]{many_chefs}
        \end{minipage}
    \end{frame}
    \begin{frame}{Processes vs Threads}
        \begin{itemize}
            \item We've talked about processes and threads individually
            \item Made the analogy of a process as the kitchen and threads as the chef in the kitchen
            \item But I don't think the example really demonstrates the tradeoffs:
            \begin{itemize}
                \item Speed
                \item Memory
                \item Bugs
            \end{itemize}
            \item I'm going to demonstrate the difference between processes and threads in practice
        \end{itemize}
    \end{frame}

    \begin{frame}{Digression: Measuring Performance}
        \begin{itemize}
            \item Measuring is nuanced, difficult to get right and is a distinct area of study
            \item My measurements here are very approximate
            \item This works because the sizes are massive
            \item \textbf{Don't do this in the real world}
        \end{itemize}
    \end{frame}

    \begin{frame}{Threads are much quicker to create}
        \begin{itemize}
            \item A process needs to manage its own memory and resources
            \item $\implies$ the process of spawning (or creating) a process is much slower than spawning a thread
            \item Let's test this
        \end{itemize}
    \end{frame}

    \begin{frame}{Speed benchmark setup}
        \lstinputlisting{code/timed.py}
    \end{frame}

    \begin{frame}{Speed benchmark setup}
        \lstinputlisting{code/time_thread.py}
    \end{frame}

    \begin{frame}{Speed benchmark setup}
        \lstinputlisting{code/time_process.py}
    \end{frame}

    \begin{frame}{Threads use much less memory}
        \begin{itemize}
            \item Creating copies of memory requires more memory (\textit{duh!})
            \item Threads do not create copies of memory while processes do
            \item Threads therefore use much less memory, but how much?
            \item Let's find out!
        \end{itemize}
    \end{frame}

    \begin{frame}{Speed benchmark setup}
        \lstinputlisting{code/memory_thread.py}
    \end{frame}

    \begin{frame}{Speed benchmark setup}
        \lstinputlisting{code/memory_process.py}
    \end{frame}

    \begin{frame}{Threads look great! Why would I ever want to use a process?}
        \begin{itemize}
            \item If a thread crashes all threads in the process will (usually) also crash
            \item There are many, very difficult to debug bugs that arise from shared memory
            \begin{itemize}
                \item Race Conditions
                \item Deadlocks
                \item Livelocks
                \item Thread Starvation
                \item Priority Starvation
                \item Memory consistency errors
                \item Atomicity violations
            \end{itemize}
        \end{itemize}
    \end{frame}

    \begin{frame}{Conclusion}
        I hope that you:
        \begin{itemize}
            \item enjoyed this session
            \item learnt something about processes and threads
            \item remember that you won't be an expert (yet!)
        \end{itemize}
    \end{frame}
\end{document}
